%        File: hw2.tex
%     Created: Mon Sep 26 04:00 PM 2016 C
% Last Change: Mon Sep 26 04:00 PM 2016 C
%

\documentclass[a4paper]{article}

\title{CSci 5421 Homework 2}
\date{10/5/16}
\author{Trevor Steil}

\usepackage{amsmath}
\usepackage{amsthm}
\usepackage{amssymb}
\usepackage{esint}
\usepackage{algorithmicx}
\usepackage{algpseudocode}
\usepackage{algorithm}

\newtheorem{theorem}{Theorem}[section]
\newtheorem{corollary}{Corollary}[section]
\newtheorem{proposition}{Proposition}[section]
\newtheorem{lemma}{Lemma}[section]
\newtheorem*{claim}{Claim}
\newtheorem*{problem}{Problem}
%\newtheorem*{lemma}{Lemma}
\newtheorem{definition}{Definition}[section]

\newcommand{\R}{\mathbb{R}}
\newcommand{\N}{\mathbb{N}}
\newcommand{\C}{\mathbb{C}}
\newcommand{\Z}{\mathbb{Z}}
\newcommand{\supp}[1]{\mathop{\mathrm{supp}}\left(#1\right)}
\newcommand{\lip}[1]{\mathop{\mathrm{Lip}}\left(#1\right)}
\newcommand{\curl}{\mathrm{curl}}
\newcommand{\la}{\left \langle}
\newcommand{\ra}{\right \rangle}
\renewcommand{\vec}[1]{\mathbf{#1}}

\newenvironment{solution}[1][]{\emph{Solution #1}}

\algnewcommand{\Or}{\textbf{ or }}
\algnewcommand{\And}{\textbf{ and }}

\begin{document}
\maketitle

\begin{enumerate}
  \item
    \begin{problem}
      Use the \textit{bottom-up} (i.e., iterative) algorithm \texttt{MATRIX-CHAIN-ORDER(p)} seen in class to determine the minimum number of
      multiplications needed to compute the product of a sequence of six matrices whose dimensions are $p = \la p_0, p_1, \dots, p_6 \ra = \la 30, 1,
      40, 10, 25, 50, 5 \ra$. You must show your work, i.e., the filled-in lookup table, the optimal parenthesization, and its cost.
    \end{problem}

    \begin{solution}

    We will fill in the lookup table according to values of $\delta = j-i$. For $\delta=0$ we have $m_{ij} = 0$ for all $i=j$.
    \vspace{-.2cm}
    \begin{equation*}
      \begin{aligned}[c]
      &\delta = 1 \\
      &\quad i=1, j=2 \\
      &\quad \quad k=1 \\
      &\quad \quad \quad 30*1*40 = 1200 \\
      &\quad m_{12} = 1200\\
      &\quad i=2, j=3 \\
      &\quad \quad k=2 \\
      &\quad \quad \quad 1 * 40 * 10 = 400 \\
      &\quad m_{23} = 400 \\
      &\quad i=3, j=4 \\
      &\quad \quad k=3 \\
      &\quad \quad \quad 40 * 10 * 25 = 10000 \\
      &\quad m_{34} = 10000 \\
      &\quad i=4, j=5 \\
      &\quad \quad k=4 \\
      &\quad \quad \quad 10 * 25 * 50 = 12500 \\
      &\quad m_{45} = 12500 \\
      &\quad i=5, j=6 \\
      &\quad \quad k=5 \\
      &\quad \quad \quad 25*50*5=6250 \\
      &\quad m_{56} = 6250
    \end{aligned}
    \quad \quad
    \begin{aligned}[c]
      &\delta=2 \\
      &\quad i=1, j=3 \\
      &\quad \quad k=1 \\
      &\quad \quad \quad 400 + 30*1*10 = 700 \\
      &\quad \quad k=2 \\
      &\quad \quad \quad 1200 + 30*40*10 = 13200 \\
      &\quad m_{13} = 700 \\
      &\quad i=2, j=4 \\
      &\quad \quad k=2 \\
      &\quad \quad \quad 10000+1*40*25=11000 \\
      &\quad \quad k=3 \\
      &\quad \quad \quad 400+1*10*25=650 \\
      &\quad m_{24}=650 \\
      &\quad i=3, j=5 \\
      &\quad \quad k=3 \\
      &\quad \quad \quad 12500+40*10*50=32500 \\
      &\quad \quad k=4 \\
      &\quad \quad \quad 10000+40*25*50=60000 \\
      &\quad m_{35} = 32500 \\
      &\quad i=4, j=6 \\
      &\quad \quad k=4 \\
      &\quad \quad \quad 6250+10*25*5=7500 \\
      &\quad \quad k=5 \text{ gives a bigger result} \\
      &\quad m_{46}=7500
    \end{aligned}
  \end{equation*}

  \begin{equation*}
    \begin{aligned}[c]
      &\delta=3 \\
      &\quad i=1, j=4 \\
      &\quad \quad k=1 \\
      &\quad \quad \quad 650 + 30*1*25=1400 \\
      &\quad \quad k=2 \text{ gives a bigger result} \\
      &\quad \quad k=3 \\
      &\quad \quad \quad 700+30*10*25=8200 \\
      &\quad m_{14} = 1400 \\
      &\quad i=2, j=5 \\
      &\quad \quad k=2 \\
      &\quad \quad \quad 32500+1*40*50=34500 \\
      &\quad \quad k=3 \\
      &\quad \quad \quad 400+12500+1*10*50=13400 \\
      &\quad \quad k=4 \\
      &\quad \quad \quad 650+1*25*50=1900 \\
      &\quad m_{25}=1900
      &\quad i=3, j=6 \\
      &\quad \quad k=3 \\
      &\quad \quad \quad 7500+40*10*5=9500 \\
      &\quad \quad k=4 \\
      &\quad \quad \quad 10000+6250+40*25*5=17500 \\
      &\quad \quad k=5 \text{ gives a bigger result} \\
      &\quad m_{36}=9500
    \end{aligned}
    \quad \quad
    \begin{aligned}[c]
      &\delta=4 \\
      &\quad i=1, j=5 \\
      &\quad \quad k=1 \\
      &\quad \quad \quad 1900+1*40*50=3900 \\
      &\quad \quad k=2 \text{ gives a bigger result} \\
      &\quad \quad k=3 \text{ gives a bigger result} \\
      &\quad \quad k=4 \\
      &\quad \quad \quad 1400+30*25*50=38900 \\
      &\quad m_{15} = 3900 \\
      &\quad i=2, j=6 \\
      &\quad \quad k=2 \\
      &\quad \quad \quad 9500+1*40*5=9700 \\
      &\quad \quad k=3 \\
      &\quad \quad \quad 400+7500+1*10*5=7950 \\
      &\quad \quad k=4 \\
      &\quad \quad \quad 650+6250+1*25*5=7025 \\
      &\quad \quad k=5 \\
      &\quad \quad \quad 1900+1*50*5=2150 \\
      &\quad m_{26}=2150
    \end{aligned}
  \end{equation*}

  \begin{equation*}
    \begin{aligned}[c]
      &\delta=5 \\
      &\quad i=1, j=6 \\
      &\quad \quad k=1 \\
      &\quad \quad \quad 2150 + 30*1*5=2300 \\
      &\quad \quad k=2 \text{ gives a bigger result} \\
      &\quad \quad k=3 \text{ gives a bigger result} \\
      &\quad \quad k=4 \text{ gives a bigger result} \\
      &\quad \quad k=5 \text{ gives a bigger result} \\
      &\quad m_{16} = 2300
    \end{aligned}
    \hspace{6.7cm}
  \end{equation*}

  This gives the following lookup table.

    \[
    \begin{bmatrix}

      0 & 1200 & 700 & 1400 & 3900 & 2300 \\
      * & 0 & 400 & 650 & 1900 & 2150 \\
      * & * & 0 & 10000 & 32500 & 9500 \\
      * & * & * & 0 & 12500 & 7500 \\
      * & * & * & * & 0 & 6250 \\
      * & * & * & * & * & 0

    \end{bmatrix}
    \]

    Therefore, the optimal parenthesization requires 2300 scalar multiplications and is given as $A_1 * ((( (A_2 * A_3) * A_4) * A_5) * A_6)$.

  \end{solution}

  \item
    \begin{problem} Ex 15.2-5
      Let $R(i,j)$ be the number of times that table entry $m[i,j]$ is referenced while computing other table entries in a call of
      \texttt{Matrix-Chain-Order}. Show that the total number of references for the entire table is
      \[ \sum_{i=1}^n \sum_{j=1}^n R(i,j) = \frac{n^3 - n}{3} .\]
    \end{problem}

    \begin{solution}[1]

      By looking at the algorithm, we can see for $j \geq i$ that $m[i,j]$ is used to calculate all values in the table to the right of position
      $(i,j)$ or above position $(i,j)$ in the lookup table. Therefore, we have
      \[ R(i,j) =
        \begin{cases}
          0 &\text{if } i > j \\
          (i-1) + (n-j) &\text{else}
        \end{cases}
      \]
      because there are $i-1$ entries above position $(i,j)$ and $n-j$ entries to the right of position $(i,j)$ in the lookup table.

      Now we can compute
      \begin{align*}
        \sum_{i=1}^n \sum_{j=1}^n R(i,j) &= \sum_{i=1}^n \sum_{j=i}^n n+i-j-1 \\
        &= \sum_{i=1}^n (n-i+1)(n+i-1) - \frac{n(n+1)}{2} + \frac{i(i-1)}{2} \\
        &= \sum_{i=1}^n n^2 - i^2 +2i - 1 -\frac{1}{2} n^2 - \frac{1}{2} n + \frac{1}{2} i^2 -\frac{1}{2} i \\
        &= \frac{1}{2} n^3 - \frac{1}{2}n^2 - n - \frac{n(n+1)(2n+1)}{12} + \frac{3n(n+1)}{4} \\
        &= \frac{1}{2} n^3 - \frac{1}{2} n^2 - n - \frac{n^3}{6} - \frac{n^2}{4} - \frac{n}{12} + \frac{3n^2}{4} + \frac{3n}{4} \\
        &= \frac{n^3}{3} - \frac{n}{3}
      \end{align*}

    \end{solution}

    \begin{solution}[2]

      An alternative (and simpler) solution is to realize that we don't have to explicitly calculate $R(i,j)$ and can instead use the structure of the
      loops in the algorithm. We see that we loop over the diagonals of our lookup table placing entries as we go. We have $\delta = j-i$ taking
      values from 1 to $n-1$. Then $i$ takes values from 1 to $n-\delta$. We assign $j = i + \delta$, and then we compute the number of
      multiplications we could get from each of the $\delta$ possible ways of splitting the product. Each of these possibilities requires looking up 2
      values from the lookup table. So we have
      \begin{align*}
        \sum_{i=1}^n \sum_{j=1}^n R(i,j) &= 2 \sum_{\delta=1}^{n-1} (n-\delta) \delta \\
        &= n^2 (n-1) - \frac{n(n-1)(2n-1)}{3} \\
        &= n^3 - n^2 - \frac{2}{3}n^3 + n^2 - \frac{1}{3}n \\
        &= \frac{n^3 - n}{3}
      \end{align*}
    \end{solution}

  \item
    \begin{problem}
      Give a top-down, memoized version of the algorithm \texttt{LCS-LENGTH(X,Y)} to compute, in $O(mn)$ time, the length of a longest common
      subsequence of strings $X$ and $Y$, where $m=|X|$ and $n=|Y|$. (You do not have to retrieve the \texttt{LCS} itself; just compute its length.)
      Give a careful analysis of the running time.
    \end{problem}

    \begin{solution}

      We will let our strings be given by $X = x_1 x_2 \dots x_m$ and $Y = y_1 y_2 \dots y_n$. We give the algorithm as

      \begin{algorithm}
        \begin{algorithmic}[1]
        \Function{Memoized-LCS}{$X,Y$}
          \For{ $i \gets 1 .. m$ }
          \For{ $j \gets 1 .. n$ }
          \State $l[i,j] \gets \infty$
          \EndFor
          \EndFor

          \State \texttt{Lookup-LCS}$(X,Y,length(X), length(Y))$
        \EndFunction
        \end{algorithmic}

        \vspace{1cm}

        \begin{algorithmic}[1]
        \Function{Lookup-LCS}{$X,Y,i,j$}
          \If{ $l[i,j] < \infty$ }
          \Return $l[i,j]$
          \Else
          \If{ $i=0 \Or j=0$ }
          \State $l[i,j] \gets 0$
          \ElsIf{ $x_i = x_j $}
          \State $l[i,j] \gets \texttt{Lookup-LCS}(X,Y,i-1,j-1)$
          \ElsIf{ $l[i-1,j] \geq l[i,j-1]$ }
          \State $l[i,j] \gets \texttt{Lookup-LCS}(X,Y,i-1,j)$
          \Else
          \State $l[i,j] \gets \texttt{Lookup-LCS}(X,Y,i,j-1)$
          \EndIf
          \EndIf
          \State
          \Return $l[i,j]$
        \EndFunction
        \end{algorithmic}
      \end{algorithm}

      We can divide calls to \texttt{Lookup-LCS} into Type 1 calls in which $l[i,j]$ must be computed and Type 2 calls in which the value of $l[i,j]$
      is retrieved from the table.

      For each entry in the table, there can be at most one Type 1 call because after the value of $l[i,j]$ is computed, it is stored in the table for
      future lookups. To compute $l[i,j]$ previous values of $l[i,j]$ are looked up and compared, both of which take $O(1)$ time. Because there are
      $mn$ entries in the lookup table, we get a contribution of $O(mn)$ to our running-time from Type 1 calls to \texttt{Lookup-LCS}.

      When trying to find the value of $l[i,j]$, \texttt{Lookup-LCS} is only called to lookup values for $l[i-1,j], l[i,j-1]$, and $l[i-1,j-1]$.
      Thinking of this instead to tell us where $Lookup-LCS(X,Y,i,j)$ can be called from, we see the only possible locations in the table to make such
      a call would have to come from positions directly below $l[i,j]$, directly right of $l[i,j]$, or one entry right and one entry down from
      $l[i,j]$. Because of this, there can be at most 3 Type 2 calls performed on any entry in our lookup table (there can actually only be 2 because
      the first call would have to be a Type 1 call). Each Type 2 call requires $O(1)$ time, so the Type 2 calls contribute a running time of $O(mn)$
      to the total running time.

      Combining the running times from Type 1 and Type 2 calls to \texttt{Lookup-LCS}, we see the running time of \texttt{Memoized-LCS} is $O(mn)$.

    \end{solution}

  \item
    \begin{problem}
      This problem explores an improvement in the $\Theta(n^3)$ running time of the algorithm \texttt{OPTIMAL-BST} (Sec 15.5). It can be shown that
      the optimal root, $root_{ij}$, satisfies $root_{i,j-1} \leq root_{ij} \leq root_{i+1,j}$, $1 \leq i < j \leq n$. (You may assume this result
      without proof). Using this result, rewrite the algorithm \texttt{OPTIMAL-BST} and prove carefully that it runs in time $\Theta(n^2)$.
    \end{problem}

    \begin{solution}

      The algorithm we write will only slightly modify the \texttt{Optimal-BST} algorithm in the book. We can use the information of previous roots by
      storing another matrix containing the roots used in each step of the calculation. We need to know how to initialize this matrix. We see that for
      a tree consisting only of the node $k_i$ and its associated dummy nodes, the root node must be $k_i$. So along the diagonal, we will have
      $r[i,i] = i$. This initialization is different than the initialization from $e$ and $w$, so it will be seen as an \texttt{if} statement in the
      pseudocode. Our algorithm is then the following:

      \begin{algorithm}
        \begin{algorithmic}[1]

          \Function{Optimal-BST}{$p,q,n$}

          \For{ $i \gets 1.. n+1$ }
          \State $e[i,i-a] \gets q_{i-1}$
          \State $w[i,i-1] \gets q_{i-1}$
          \EndFor

          \For{ $l \gets 1..n$ }
          \For{ $i \gets 1 .. n-l+1$}
          \State $j \gets i+l-1$
          \State $e[i,j] \gets \infty$
          \State $w[i,j] \gets w[i,j-1] + p_j + q_j$
          \If{ $i=j$ }
          \State $e[i,i] \gets e[i,i-1] + e[i+1,i] + w[i,i]$
          \State $root[i,i] = i$
          \Else
          \For{ $r \gets root[i,j-1] .. root[i+1,j]$ }
          \State $t \gets e[i,r-1] + e[r+1,j] + w[i,j]$
          \If{ $t < e[i,j]$ }
          \State $e[i,j] \gets t$
          \State $root[i,j] \gets r$
          \EndIf
          \EndFor
          \EndIf
          \EndFor
          \EndFor

          \EndFunction
        \end{algorithmic}
      \end{algorithm}

      We see from the pseudocode that the innermost loop runs at most $root[i+1,j] - root[i,j-1] + 1$ times. We do not know exactly how many times
      this is, but we can add up this time for all $i$ with a fixed $l$.

      Let $l \geq 2$. Then $j= i+l-1$. The largest number of times the inner two loops can run is given by
      \begin{align*}
        \sum_{i=1}^{n-l+1} r[i+1,i+l-1] - r[i,i_l-2] + 1 &= r[n-l+2,n] - r[1,l-1] + n-l+1 \quad \parbox{3cm}{because the sum is telescoping} \\
        &\leq 2n - l \\
        &= O(n)
      \end{align*}
      which is less than the $O(n^2)$ we get letting our innermost loop run from $i$ to $j$.
      The outer loop runs for $l=1 .. n$, giving us another $n$ in our overall running time. Therefore, we have an overall running time of $O(n^2)$.

    \end{solution}

  \item
    \begin{problem}
      Ex. 25.2-4 Justify your answers carefully

      As it appears in the text, the Floyd-Warshall algorithm requires $\Theta(n^3)$ space, since we compute $d_{ij}^{(k)}$ for $i,j,k =1, 2, \dots,
      n$. Show that the following procedure, which simply drops all the superscripts, is correct, and thus only $\Theta(n^2)$ space is required

      \begin{algorithmic}[1]
        \Function{Floyd-Warshall'}{$W$}

        \State $n \gets rows[W]$
        \State $D \gets W$

        \For{ $k \gets 1.. n$ }
        \For{ $i \gets 1..n$ }
        \For{ $j \gets 1..n$ }
        \State $d_{ij} \gets \min( d_{ij}, d_{ik} + d_{kj})$
        \EndFor
        \EndFor
        \EndFor

        \Return $D$
        \EndFunction
      \end{algorithmic}
    \end{problem}

  \item
    \begin{problem}

      Consider the problem of transforming a string, $A = a_1 a_2 \dots a_m$, of characters into another string, $B=b_1 b_2 \dots b_n$, by a sequence
      of insert(I), delete(D), and substitute(S) operations.

      Our goal is to come up with a transformation that minimizes the total number of I,D, and S operations. Define the edit distance between two
      strings as the minimum number of I,D, and S operations needed to transform the first string into the second. Design a bottom-up dynamic
      programming algorithm to compute the edit distance between $A$ and $B$ in time $\Theta(mn)$. The output is the edit distance and the
      corresponding operations.

      Your answer should include (i) a brief description of the main ideas, including the recurrence equation and its justifications, (ii) pseudocode,
      and (iii) an analysis of the running time.

      Hint: Let $e(i,j)$ be the edit distance between $a_1, \dots, a_i$ and $b_1, \dots, b_j$.

    \end{problem}

    \begin{solution}

      The main idea for this problem is understanding how optimal substructures can be used. Define $A_i = a_1 a_2 \dots a_i$ and $B_j = b_1 b_2 \dots
      b_j$. We will be constructing a lookup table with entries
      $e[i,j]$ for $i=1, \dots, m$ and $j = 1, \dots, n$, where $e[i,j]$ is the edit distance between $A_i$ and $B_j$.

      If $a_i = b_j$, we see that no extra edits from changing $A_{i-1}$ to $B_{j-1}$ are necessary to change $A_i$ to $B_j$. Therefore, $e[i,j] \leq
      e[i-1,j-1]$ in this case. If $e[i,j]<e[i-1,j-1]$ we would have a shorter sequence of edits from $A_{i-1}$ to $B_{j-1}$ by ignoring the match of
      the last entries of each string, contradicting the definition of $e[i-1,j-1]$. Therefore, if $a_i = b_j$, then $e[i,j] = e[i-1,j-1]$.

      If $a_i \neq b_j$, we can still construct the corresponding set of edits from the edit sequences of converting substrings. The sequence
      corresponding to $e[i-1,j-1]$ can have a single substitution added to get an edit sequence to change $A_i$ to $B_j$. An edit sequence
      corresponding to $e[i-1,j]$ can have a single insertion added to get a sequence to change $A_{i}$ to $B_j$. An edit sequence corresponding to
      $e[i,j-1]$ can have a single deletion added to get a sequence to change $A_i$ to $B_j$. We can also see that taking an edit sequence to change
      $A_i$ to $B_j$ and removing the substitution, insertion, or deletion that modifies the furthest character to the right, we get an edit sequence
      corresponding to changing $A_{i-1}$ to $B_{j-1}$, $A_{i-1}$ to $B_j$, or $A_i$ to $B_{j-1}$.

      This also lets us see how the entries $e[i,j]$ can be computed. We get the relation
      \[ e[i,j] =
      \begin{cases}
        e[i-1,j-1] &\text{if } a_i = b_j \\
        \min \{ e[i-1,j-1], e[i-1,j], e[i,j-1] \} + 1 &\text{else}
      \end{cases}
      \]

      By considering $A$ or $B$ being the empty string, we see that the first row and column can be initialized to
      \[ e[0,j] = j \]
      and
      \[ e[i,0] = i \]
      because starting or ending at the empty string requires all characters to either be inserted or deleted.

      Putting this into an algorithm gives
      \vspace{.5cm}
      \begin{algorithm}
        \begin{algorithmic}[1]
          \Function{Edit-Distance}{$A,B$}

          \State $m \gets length(A)$
          \State $n \gets length(B)$

          \For{ $i \gets 0 .. m$ }
          \State $e[i,0] = i$
          \EndFor

          \For{ $j \gets 0 .. n$ }
          \State $e[0,j] = j$
          \EndFor

          \For{ $i \gets 1..m$ }
          \For{ $j \gets 1..n$ }
          \If{ $a_i = b_j$ }
          \State $e[i,j] \gets e[i-1,j-1]$
          \State $p[i,j] \gets ``\nwarrow"$
          \Else
          \If{ $e[i-1,j-1] \leq e[i, j-1] \And e[i-1,j-1] \leq e[i-1,j]$ }
          \State $e[i,j] \gets e[i-1,j-1] + 1$
          \State $p[i,j] \gets ``\nwarrow"$
          \ElsIf{ $e[i,j-1] \leq e[i-1,j]$ }
          \State $e[i,j] \gets e[i,j-1]+1$
          \State $p[i,j] \gets ``\leftarrow"$
          \Else
          \State $e[i,j] \gets e[i-1,j]+1$
          \State $p[i,j] \gets ``\uparrow"$
          \EndIf
          \EndIf
          \EndFor
          \EndFor

          \hspace{-.25cm} \Return $e \And p$

          \EndFunction
        \end{algorithmic}
      \end{algorithm}

      This generates a matrix where the edit distance between $A$ and $B$ is contained in $e[m,n]$. To obtain the sequence of edits necessary to
      change $A$ to $B$, we must use the extra information stored in $p$. We do this by starting at the lower-right corner and traversing the arrows.
      Deletions correspond to $``\uparrow"$. Insertions correspond to $``\leftarrow"$. Substitutions and matchings correspond to $``\nwarrow"$. We
      are able to distinguish between substitutions and matchings because $e[i,j] = e[i-1,j-1]$ for a matching, but $e[i,j] = e[i-1,j-1] + 1$ for a
      substitution. We print this edit sequence through recursive calls to a print function.

      \begin{algorithm}
        \begin{algorithmic}[1]
          \Function{Print-Edit-Sequence}{$A,B,i,j$}

          \If{ $p[i,j] = ``\nwarrow"$ }
          \If{ $e[i,j] = e[i-1,j-1]$ }
          \State \texttt{Print-Edit-Sequence}$(A,B,i-1,j-1)$
          \State \texttt{print} ``M''
          \Else
          \State \texttt{Print-Edit-Sequence}$(A,B,i-1,j-1)$
          \State \texttt{print} ``S''
          \EndIf
          \ElsIf{ $p[i,j] = ``\leftarrow"$ }
          \State \texttt{Print-Edit-Sequence}$(A,B,i,j-1)$
          \State \texttt{print} ``I''
          \Else
          \State \texttt{Print-Edit-Sequence}$(A,B,i-1,j)$
          \State \texttt{print} ``D''
          \EndIf

          \EndFunction
        \end{algorithmic}
      \end{algorithm}

      In \texttt{Edit-Distance}, we have nested loops containing operations that require $O(1)$ time, so this contributes $O(mn)$ to the run time. In
      \texttt{Print-Edit-Sequence}, there are recursive calls that decrease $i$ or $j$ in every call, and only operations that are $O(1)$ are
      performed. So \texttt{Print-Edit-Sequence} has a running time of $O(m+n)$. This gives an overall running time of $O(mn)$.

    \end{solution}

  \item
    \begin{problem}
      Consider the following multiplication table defined on an alphabet $\Sigma = \{a, b\}$.

      \[
        \begin{matrix}
          \ & | & a & b \\
          -- & | & -- & -- \\
          a & | & b & a \\
          b & | & b & b
        \end{matrix}
      \]

      The rows correspond to the left operand and the columns to the right operand; thus, $aa = b, ab = a,$ etc.

      Design a bottom-up dynamic programming algorithm which takes a string $X = x_1 x_2 \dots x_n$, where each $x_i \in \Sigma$, and outputs ``true''
      if there is a parenthesization of $X$ for which the expression evaluates to $a$ (under the above multiplication table) and ``false'' otherwise.
      (You do not have to compute the parenthesization itself if the output is ``true''). The target time bound is $\Theta(n^3)$.

      Your answer should include (i) a brief description of the main ideas, including the recurrence equation and its justification, (ii) pseudocode,
      and (iii) an analysis of the running time.

      Note: This is a decision problem, not an optimization problem. Such problems can also be solved sometimes via dynamic programming.

      Hint: Let $a_{ij}$ (resp. $b_{ij}$) be ``true'' if there is a parenthesization of $x_1 x_2 \dots x_j$ which evaluates to $a$ (res. $b$) and
      ``false'' otherwise.

    \end{problem}

    \begin{solution}

      We see from the multiplication table that the only way for a nontrivial product to have some parenthesization that results in $a$ is for there
      to be some point the product can be split where left of the split can be parenthesized to give $a$ and right of the split can be parenthesized
      to give $b$. We must therefore keep track of whether parenthesizations exist that result in $a$ as well as whether parenthesizations exist that
      result in $b$.

      Let $X=x_1 x_2 \dots x_n$ be our string. Define $A = (a_{ij})$ where $a_{ij} =$ ``true'' if there exists a parenthesization of $x_i \dots x_j$ that results in
      $a$, and $a_{ij}=$ ``false'' otherwise. Similarly, define $B = (b_{ij})$ where $b_{ij} =$ ``true'' if there exists a parenthesization of $x_i \dots x_j$ that results in $b$, and
      $b_{ij}=$ ``false'' otherwise.

      From the multiplication table, we have $a_{ij} =$ ``true'' if there is some $i \leq k < j$ such that $a_{ik}=$ ``true'' and $b_{(k+1)j}=$
      ``true''. Similarly, $b_{ij} =$ ``true'' if there is some $i \leq k < j$ such that $b_{ik}=$ ``true'' or if there is some $i \leq k < j$ such
      that $a_{ik}=$ ``true'' and $a_{(k+1)j} =$ ``true''.

      The lookup tables are structured and filled very similarly to those for Matrix Chain Multiplication. We only use the upper triangular portion of
      our matrices. In the end, we want the value $a_{1n}$. We fill the matrix along its diagonals. The main diagonal can be filled by
      \[ a[i,i] =
      \begin{cases}
        \text{``true"} &\text{if } x_i = a \\
        \text{``false"} &\text{else}
      \end{cases}
      \]
      and similarly for $b[i,i]$.

      This gives us the following algorithm:

      \begin{algorithm}
        \begin{algorithmic}[1]
          \Function{Multiplication-Table}{$X$}

          \State $n \gets length(X)$
          \For{ $i \gets 1..n$ }
          \For{ $j \gets i..n$ }
          \State $a[i,j] \gets \text{``false"}$
          \State $b[i,j] \gets \text{``false"}$
          \EndFor
          \EndFor

          \For{ $i \gets 1..n$ }
          \If{ $x_i = a$ }
          \State $a[i,i] \gets \text{``true"}$
          \Else
          \State $b[i,i] \gets \text{``true"}$
          \EndIf
          \EndFor

          \For{ $\delta \gets 1..n-1$ }
          \For{ $i \gets 1..n-\delta$ }
          \State $j \gets i + \delta$
          \For{ $k \gets i..j-1$ }
          \If{ $a[i,k] = \text{``true''}$ }
          \If{ $b[k+1,j] = \text{``true''}$ }
          \State $a[i,j] \gets \text{``true''}$
          \ElsIf{ $a[k+1,j] = \text{``true''}$ }
          \State $b[i,j] \gets \text{``true''}$
          \EndIf
          \Else
          \State $b[i,j] \gets \text{``true''}$
          \EndIf
          \EndFor
          \EndFor
          \EndFor

          \hspace{-.25cm} \Return $a[1,n]$

          \EndFunction
        \end{algorithmic}
      \end{algorithm}

      All of the operations performed here are comparisons and assignments that take $O(1)$ time each. Each of the 3 loops is running over at most $n$
      values, so the overall running time of this algorithm is $O(n^3)$. A slightly more careful analysis would see our innermost loop is running over
      $\delta$ items, giving a running time of $O(\delta)$. The next loop out from there is running over $n-\delta$ values, giving a cumulative
      running time of $O(\delta(n-\delta))$ when combined with the inner loop. The outermost loop is then looping over all values of $\delta$ from 1
      to $n$. So the overall running time is
      \begin{align*}
        O \left( \sum_{\delta=1}^n \delta(n-\delta) \right) &= O \left( \sum_{\delta=1}^n \delta^2 \right) \\
        &= O \left( \frac{n(n+1)(2n+1)}{6} \right) \\
        &= O(n^3)
      \end{align*}
      exactly as our initial analysis showed.

    \end{solution}

\end{enumerate}
\end{document}


