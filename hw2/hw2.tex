%        File: hw2.tex
%     Created: Mon Sep 26 04:00 PM 2016 C
% Last Change: Mon Sep 26 04:00 PM 2016 C
%

\documentclass[a4paper]{article}

\title{CSci 5421 Homework 2}
\date{10/5/16}
\author{Trevor Steil}

\usepackage{amsmath}
\usepackage{amsthm}
\usepackage{amssymb}
\usepackage{esint}

\newtheorem{theorem}{Theorem}[section]
\newtheorem{corollary}{Corollary}[section]
\newtheorem{proposition}{Proposition}[section]
\newtheorem{lemma}{Lemma}[section]
\newtheorem*{claim}{Claim}
\newtheorem*{problem}{Problem}
%\newtheorem*{lemma}{Lemma}
\newtheorem{definition}{Definition}[section]

\newcommand{\R}{\mathbb{R}}
\newcommand{\N}{\mathbb{N}}
\newcommand{\C}{\mathbb{C}}
\newcommand{\Z}{\mathbb{Z}}
\newcommand{\supp}[1]{\mathop{\mathrm{supp}}\left(#1\right)}
\newcommand{\lip}[1]{\mathop{\mathrm{Lip}}\left(#1\right)}
\newcommand{\curl}{\mathrm{curl}}
\newcommand{\la}{\left \langle}
\newcommand{\ra}{\right \rangle}
\renewcommand{\vec}[1]{\mathbf{#1}}

\newenvironment{solution}{\emph{Solution.}}

\begin{document}
\maketitle

\begin{enumerate}
  \item
    \begin{problem}
      Use the \textit{bottom-up} (i.e., iterative) algorithm \texttt{MATRIX-CHAIN-ORDER(p)} seen in class to determine the minimum number of
      multiplications needed to compute the product of a sequence of six matrices whose dimensions are $p = \la p_0, p_1, \dots, p_6 \ra = \la 30, 1,
      40, 10, 25, 50, 5 \ra$. You must show your work, i.e., the filled-in lookup table, the optimal parenthesization, and its cost.
    \end{problem}

    We will fill in the lookup table according to values of $\delta = j-i$. For $\delta=0$ we have $m_{ij} = 0$.
    \begin{equation*}
    %\begin{flalign*}
      \begin{aligned}[c]
      &\delta = 1 \\
      &\quad i=1, j=2 \\
      &\quad \quad k=1 \\
      &\quad \quad \quad 30*1*40 = 1200 \\
      &\quad m_{12} = 1200\\
      &\quad i=2, j=3 \\
      &\quad \quad k=2 \\
      &\quad \quad \quad 1 * 40 * 10 = 400 \\
      &\quad m_{23} = 400 \\
      &\quad i=3, j=4 \\
      &\quad \quad k=3 \\
      &\quad \quad \quad 40 * 10 * 25 = 10000 \\
      &\quad m_{34} = 10000 \\
      &\quad i=4, j=5 \\
      &\quad \quad k=4 \\
      &\quad \quad \quad 10 * 25 * 50 = 12500 \\
      &\quad m_{45} = 12500 \\
      &\quad i=5, j=6 \\
      &\quad \quad k=5 \\
      &\quad \quad \quad 25*50*5=6250 \\
      &\quad m_{56} = 6250
    %\end{flalign*}
    \end{aligned}
    \quad \quad
    \begin{aligned}[c]
      &\delta=2 \\
      &\quad i=1, j=3 \\
      &\quad \quad k=1 \\
      &\quad \quad \quad 400 + 30*1*10 = 700 \\
      &\quad \quad k=2 \\
      &\quad \quad \quad 1200 + 30*40*10 = 13200 \\
      &\quad m_{13} = 700 \\
      &\quad i=2, j=4 \\
      &\quad \quad k=2 \\
      &\quad \quad \quad 10000+1*40*25=11000 \\
      &\quad \quad k=3 \\
      &\quad \quad \quad 400+1*10*25=650 \\
      &\quad m_{24}=650 \\
      &\quad i=3, j=5 \\
      &\quad \quad k=3 \\
      &\quad \quad \quad 12500+40*10*50=32500 \\
      &\quad \quad k=4 \\
      &\quad \quad \quad 10000+40*25*50=60000 \\
      &\quad m_{35} = 32500 \\
      &\quad i=4, j=6 \\
      &\quad \quad k=4 \\
      &\quad \quad \quad 6250+10*25*5=7500 \\
      &\quad \quad k=5 \text{ gives a bigger result} \\
      &\quad m_{46}=7500
    \end{aligned}
  \end{equation*}

  \begin{equation*}
    \begin{aligned}[c]
      &\delta=3 \\
      &\quad i=1, j=4 \\
      &\quad \quad k=1 \\
      &\quad \quad \quad 650 + 30*1*25=1400 \\
      &\quad \quad k=2 \text{ gives a bigger result} \\
      &\quad \quad k=3 \\
      &\quad \quad \quad 700+30*10*25=8200 \\
      &\quad m_{14} = 1400 \\
      &\quad i=2, j=5 \\
      &\quad \quad k=2 \\
      &\quad \quad \quad 32500+1*40*50=34500 \\
      &\quad \quad k=3 \\
      &\quad \quad \quad 400+12500+1*10*50=13400 \\
      &\quad \quad k=4 \\
      &\quad \quad \quad 650+1*25*50=1900 \\
      &\quad m_{25}=1900
      &\quad i=3, j=6 \\
      &\quad \quad k=3 \\
      &\quad \quad \quad 7500+40*10*5=9500 \\
      &\quad \quad k=4 \\
      &\quad \quad \quad 10000+6250+40*25*5=17500 \\
      &\quad \quad k=5 \text{ gives a bigger result} \\
      &\quad m_{36}=9500
    \end{aligned}
    \quad \quad
    \begin{aligned}[c]
      &\delta=4 \\
      &\quad i=1, j=5 \\
      &\quad \quad k=1 \\
      &\quad \quad \quad 1900+1*40*50=3900 \\
      &\quad \quad k=2 \text{ gives a bigger result} \\
      &\quad \quad k=3 \text{ gives a bigger result} \\
      &\quad \quad k=4 \\
      &\quad \quad \quad 1400+30*25*50=38900 \\
      &\quad m_{15} = 3900 \\
      &\quad i=2, j=6 \\
      &\quad \quad k=2 \\
      &\quad \quad \quad 9500+1*40*5=9700 \\
      &\quad \quad k=3 \\
      &\quad \quad \quad 400+7500+1*10*5=7950 \\
      &\quad \quad k=4 \\
      &\quad \quad \quad 650+6250+1*25*5=7025 \\
      &\quad \quad k=5 \\
      &\quad \quad \quad 1900+1*50*5=2150 \\
      &\quad m_{26}=2150
    \end{aligned}
  \end{equation*}

  \begin{equation*}
    \begin{aligned}[c]
      &\delta=5 \\
      &\quad i=1, j=6 \\
      &\quad \quad k=1 \\
      &\quad \quad \quad 2150 + 30*1*5=2300 \\
      &\quad \quad k=2 \text{ gives a bigger result} \\
      &\quad \quad k=3 \text{ gives a bigger result} \\
      &\quad \quad k=4 \text{ gives a bigger result} \\
      &\quad \quad k=5 \text{ gives a bigger result} \\
      &\quad m_{16} = 2300
    \end{aligned}
    \hspace{6.7cm}
  \end{equation*}

  This gives the following lookup table.

    \[
    \begin{bmatrix}

      0 & 1200 & 700 & 1400 & 3900 & 2300 \\
      * & 0 & 400 & 650 & 1900 & 2150 \\
      * & * & 0 & 10000 & 32500 & 9500 \\
      * & * & * & 0 & 12500 & 7500 \\
      * & * & * & * & 0 & 6250 \\
      * & * & * & * & * & 0

    \end{bmatrix}
    \]

    Therefore, the optimal parenthesization requires 2300 scalar multiplications and is given as $A_1 * ((( (A_2 * A_3) * A_4) * A_5) * A_6)$.

  \item
    \begin{problem} Ex 15.2-5
    \end{problem}

  \item
    \begin{problem}
      Give a top-down, memoized version of the algorithm \texttt{LCS-LENGTH(X,Y)} to compute, in $O(mn)$ time, the length of a longest common
      subsequence of strings $X$ and $Y$, where $m=|X|$ and $n=|Y|$. (You do not have to retrieve the \texttt{LCS} itself; just compute its length.)
      Give a careful analysis of the running time.
    \end{problem}

  \item
    \begin{problem}
      This problem explores an improvement in the $\Theta(n^3)$ running time of the algorithm \texttt{OPTIMAL-BST} (Sec 15.5). It can be shown that
      the optimal root, $root_{ij}$, satisfies $root_{i,j-1} \leq root_{ij} \leq root_{i+1,j}$, $1 \leq i < j \leq n$. (You may assume this result
      without proof). Using this result, rewrite the algorithm \texttt{OPTIMAL-BST} and prove carefully that it runs in time $\Theta(n^2)$.
    \end{problem}

  \item
    \begin{problem}
      Ex. 25.2-4 Justify your answers carefully
    \end{problem}

  \item
    \begin{problem}

      Consider the problem of transforming a string, $A = a_1 a_2 \dots a_m$, of characters into another string, $B=b_1 b_2 \dots b_n$, by a sequence
      of insert(I), delete(D), and substitute(S) operations.

      Our goal is to come up with a transformation that minimizes the total number of I,D, and S operations. Define the edit distance between two
      strings as the minimum number of I,D, and S operations needed to transform the first string into the second. Design a bottom-up dynamic
      programming algorithm to compute the edit distance between $A$ and $B$ in time $\Theta(mn)$. The output is the edit distance and the
      corresponding operations.

      Your answer should include (i) a brief description of the main ideas, including the recurrence equation and its justifications, (ii) pseudocode,
      and (iii) an analysis of the running time.

      Hint: Let $e(i,j)$ be the edit distance between $a_1, \dots, a_i$ and $b_1, \dots, b_j$.

    \end{problem}

  \item
    \begin{problem}
      Consider the following multiplication table defined on an alphabet $\Sigma = \{a, b\}$.

      \large{\textbf{INSERT TABLE}}

      The rows correspond to the left operand and the columns to the right operand; thus, $aa = b, ab = a,$ etc.

      Design a bottom-up dynamic programming algorithm which takes a string $X = x_1 x_2 \dots x_n$, where each $x_i \in \Sigma$, and outputs ``true''
      if there is a parenthesization of $X$ for which the expression evaluates to $a$ (under the above multiplication table) and ``false'' otherwise.
      (You do not have to compute the parenthesization itself if the output is ``true''). The target time bound is $\Theta(n^3)$.

      Your answer should include (i) a brief description of the main ideas, including the recurrence equation and its justification, (ii) pseudocode,
      and (iii) an analysis of the running time.

      Note: This is a decision problem, not an optimization problem. Such problems can also be solved sometimes via dynamic programming.

      Hint: Let $a_{ij}$ (resp. $b_{ij}$) be ``true'' if there is a parenthesization of $x_1 x_2 \dots x_j$ which evaluates to $a$ (res. $b$) and
      ``false'' otherwise.

    \end{problem}

\end{enumerate}<++>
\end{document}


