%        File: hw6.tex
%     Created: Wed Nov 30 04:00 PM 2016 C
% Last Change: Wed Nov 30 04:00 PM 2016 C
%

\documentclass[a4paper]{article}

\title{CSci 5421 Homework 6}
\date{12/14/16}
\author{Trevor Steil}

\usepackage{amsmath}
\usepackage{amsthm}
\usepackage{amssymb}
\usepackage{esint}
\usepackage{enumitem}
\usepackage{algorithm}
\usepackage{algorithmicx}
\usepackage{algpseudocode}
\usepackage{bbm}

\newtheorem{theorem}{Theorem}[section]
\newtheorem{corollary}{Corollary}[section]
\newtheorem{proposition}{Proposition}[section]
\newtheorem{lemma}{Lemma}[section]
\newtheorem*{claim}{Claim}
\newtheorem*{problem}{Problem}
%\newtheorem*{lemma}{Lemma}
\newtheorem{definition}{Definition}[section]

\newcommand{\R}{\mathbb{R}}
\newcommand{\N}{\mathbb{N}}
\newcommand{\C}{\mathbb{C}}
\newcommand{\Z}{\mathbb{Z}}
\newcommand{\Q}{\mathbb{Q}}
\newcommand{\E}{\mathbb{E}}
\newcommand{\supp}[1]{\mathop{\mathrm{supp}}\left(#1\right)}
\newcommand{\lip}[1]{\mathop{\mathrm{Lip}}\left(#1\right)}
\newcommand{\curl}{\mathrm{curl}}
\newcommand{\la}{\left \langle}
\newcommand{\ra}{\right \rangle}
\renewcommand{\vec}[1]{\mathbf{#1}}

\newenvironment{solution}[1][]{\emph{Solution #1}}

\algnewcommand{\Or}{\textbf{ or }}
\algnewcommand{\And}{\textbf{ or }}

\begin{document}
\maketitle
\begin{enumerate}
  \item
    \begin{problem}
      Suppose that a stack, $S$, has $s>0$ items on it initially and an arbitrary sequence of $n$ \texttt{PUSH} and \texttt{MULTIPOP} operations is
      executed. Use the potential method of amortized analysis to show that the total actual cost of the sequence is $O(n)$, if $n = \Omega(s)$.

      Note: You will find the discussion beginning at the bottom of page 461 helpful.
    \end{problem}

    \begin{solution}

    \end{solution}

  \item
    \begin{problem}
      Show the Fibonacci heap that results from performing a \texttt{FIB-HEAP-DELETE-MIN} operation on the Fibonacci heap, $h$, given in the
      assignment. Show intermediate steps (including those done during consolidation) and marked nodes clearly. (Marked nodes are indicated by a '*'.)
      For consistency, do consolidation starting from the root to the right of the current minimum node.
    \end{problem}

    \begin{solution}
      See attachment at end.
    \end{solution}

  \item Problem 19-1
    \begin{problem}
      \textbf{Alternative implementation of deletion}

      Professor Pisano has proposed the following variant of the \texttt{FIB-HEAP-DELETE} procedure claiming that it runs faster when the node being
      deleted is no the node pointed to by $H.min.$

      \begin{algorithmic}[1]
        \Function {Pisano-Delete}{$H,x$}

        \If{ $ x = H.min$ }
        \State \texttt{FIB-HEAP-EXTRACT-MIN}$(H)$
        \Else
        \State $y \gets x.p$
        \If{ $y \neq H.nil$ }
        \State \texttt{CUT}$(H,x,y)$
        \State \texttt{CASCADING-CUT}$(H,y)$
        \EndIf
        \State add $x$'s child list to the root list of $H$
        \State remove $x$ from the root list of $H$
        \EndIf

        \EndFunction
      \end{algorithmic}

      \begin{enumerate}
        \item The professor's claim that this procedure runs faster is based partly on the assumption that line 7 can be performed in $O(1)$ actual
          time. What is wrong with this assumption?

        \item Give a good upper bound on the actual time of \texttt{PISANO-DELETE} when $x$ is not $H.min$. Your bound should be in terms of
          $x.degree$ and the number $c$ of calls to the \texttt{CASCADING-CUT} procedure.

        \item
          Suppose that we call \texttt{PISANO-DELETE}$(H,x)$ and let $H'$ be the Fibonacci heap that results. Assuming that node $x$ is not a root,
          bound the potential of $H'$ in terms of $x.degree, c, t(H)$, and $m(H)$.

        \item
          Conclude that the amortized time for \texttt{PISANO-DELETE} is asymptotically no better than for \texttt{FIB-HEAP-DELETE}, even when $x \neq
          H.min$.
      \end{enumerate}

      Use the potential function $\varphi(H) = t(H) + 2 m(H)$.

    \end{problem}

    \begin{solution}

    \end{solution}

  \item

  \item

\end{enumerate}
\end{document}


