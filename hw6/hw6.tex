%        File: hw6.tex
%     Created: Wed Nov 30 04:00 PM 2016 C
% Last Change: Wed Nov 30 04:00 PM 2016 C
%

\documentclass[a4paper]{article}

\title{CSci 5421 Homework 6}
\date{12/14/16}
\author{Trevor Steil}

\usepackage{amsmath}
\usepackage{amsthm}
\usepackage{amssymb}
\usepackage{esint}
\usepackage{enumitem}
\usepackage{algorithm}
\usepackage{algorithmicx}
\usepackage{algpseudocode}
\usepackage{bbm}

\newtheorem{theorem}{Theorem}[section]
\newtheorem{corollary}{Corollary}[section]
\newtheorem{proposition}{Proposition}[section]
\newtheorem{lemma}{Lemma}[section]
\newtheorem*{claim}{Claim}
\newtheorem*{problem}{Problem}
%\newtheorem*{lemma}{Lemma}
\newtheorem{definition}{Definition}[section]

\newcommand{\R}{\mathbb{R}}
\newcommand{\N}{\mathbb{N}}
\newcommand{\C}{\mathbb{C}}
\newcommand{\Z}{\mathbb{Z}}
\newcommand{\Q}{\mathbb{Q}}
\newcommand{\E}{\mathbb{E}}
\newcommand{\supp}[1]{\mathop{\mathrm{supp}}\left(#1\right)}
\newcommand{\lip}[1]{\mathop{\mathrm{Lip}}\left(#1\right)}
\newcommand{\curl}{\mathrm{curl}}
\newcommand{\la}{\left \langle}
\newcommand{\ra}{\right \rangle}
\renewcommand{\vec}[1]{\mathbf{#1}}

\newenvironment{solution}[1][]{\emph{Solution #1}}

\algnewcommand{\Or}{\textbf{ or }}
\algnewcommand{\And}{\textbf{ or }}

\begin{document}
\maketitle
\begin{enumerate}
  \item
    \begin{problem}
      Suppose that a stack, $S$, has $s>0$ items on it initially and an arbitrary sequence of $n$ \texttt{PUSH} and \texttt{MULTIPOP} operations is
      executed. Use the potential method of amortized analysis to show that the total actual cost of the sequence is $O(n)$, if $n = \Omega(s)$.

      Note: You will find the discussion beginning at the bottom of page 461 helpful.
    \end{problem}

    \begin{solution}

      This is the same as the potential method starting from an empty stack with a slight modification when amortized costs are summed. As before we
      define our potential function to be
      \[ \varphi(S) = |S|, \]
      that is, our potential function is the size of the stack. We then define $\widehat{c}_i = c_i + \varphi_i - \varphi_{i-1}$ where $\widehat{c}_i$
      is the amortized cost of the $i^{th}$ operation, $c_i$ is the actual cost of the $i^{th}$ operation, and $\varphi_i$ is the potential function
      after the $i^{th}$ operation. This leads to the same amortized costs as before:
      \[ \widehat{c}_i =
        \begin{cases}
          2 &\text{ if $i^{th}$ operation is \texttt{PUSH} } \\
          0 &\text{ if $i^{th}$ operation is \texttt{MULTIPOP} .}
        \end{cases}
      \]

      Then we have
      \begin{align*}
        \sum_{i=1}^n \widehat{c}_i &= \sum_{i=1}^n (c_i + \varphi_i - \varphi_{i-1}) \\
        &= \sum_{i=1}^n c_i + \varphi_n - \varphi_0 \quad \parbox{5cm}{because of the telescoping sum} \\
        &= \sum_{i=1}^n c_i + \varphi_n - s \quad \parbox{5cm}{because stack initially contains $s$ elements}
      \end{align*}

      Because $\varphi_n \geq 0$ by definition, we get the upper bound
      \[ s + \sum_{i=1}^n \widehat{c}_i \geq \sum_{i=1}^n c_i .\]
      By definition, $\widehat{c}_i = O(1)$, so
      \[ s + O(n) \geq \sum_{i=1}^n c_i .\]
      We assumed $n = \Omega(s)$, which is equivalent to $s = O(n)$, so
      \[ \sum_{i=1}^n c_i = O(n) .\]

    \end{solution}

  \item
    \begin{problem}
      Show the Fibonacci heap that results from performing a \texttt{FIB-HEAP-DELETE-MIN} operation on the Fibonacci heap, $h$, given in the
      assignment. Show intermediate steps (including those done during consolidation) and marked nodes clearly. (Marked nodes are indicated by a '*'.)
      For consistency, do consolidation starting from the root to the right of the current minimum node.
    \end{problem}

    \begin{solution}
      See attachment at end.
    \end{solution}

  \item Problem 19-1
    \begin{problem}
      \textbf{Alternative implementation of deletion}

      Professor Pisano has proposed the following variant of the \texttt{FIB-HEAP-DELETE} procedure claiming that it runs faster when the node being
      deleted is not the node pointed to by $H.min.$

      \begin{algorithmic}[1]
        \Function {Pisano-Delete}{$H,x$}

        \If{ $ x = H.min$ }
        \State \texttt{FIB-HEAP-EXTRACT-MIN}$(H)$
        \Else
        \State $y \gets x.p$
        \If{ $y \neq H.nil$ }
        \State \texttt{CUT}$(H,x,y)$
        \State \texttt{CASCADING-CUT}$(H,y)$
        \EndIf
        \State add $x$'s child list to the root list of $H$
        \State remove $x$ from the root list of $H$
        \EndIf

        \EndFunction
      \end{algorithmic}

      \begin{enumerate}
        \item The professor's claim that this procedure runs faster is based partly on the assumption that line 10 can be performed in $O(1)$ actual
          time. What is wrong with this assumption?

        \item Give a good upper bound on the actual time of \texttt{PISANO-DELETE} when $x$ is not $H.min$. Your bound should be in terms of
          $x.degree$ and the number $c$ of calls to the \texttt{CASCADING-CUT} procedure.

        \item
          Suppose that we call \texttt{PISANO-DELETE}$(H,x)$ and let $H'$ be the Fibonacci heap that results. Assuming that node $x$ is not a root,
          bound the potential of $H'$ in terms of $x.degree, c, t(H)$, and $m(H)$.

        \item
          Conclude that the amortized time for \texttt{PISANO-DELETE} is asymptotically no better than for \texttt{FIB-HEAP-DELETE}, even when $x \neq
          H.min$.
      \end{enumerate}

      Use the potential function $\varphi(H) = t(H) + 2 m(H)$.

    \end{problem}

    \begin{solution}

      \begin{enumerate}
        \item
          Because the root list and the list of children of $x$ are implemented as doubly-linked lists, the two can be concatenated in $O(1)$ time.
          The problem is that the children of $x$ must all have their parents updated to $H.nil$, which will take $O(d)$ time, where $d = x.degree$.

        \item
          From above, adding the children of $x$ to the root list requires $O(d)$ time. Each call to \texttt{CASCADING-CUT} requires $O(1)$ time,
          excluding recursive calls. This gives an actual cost of $O(d + c)$.

        \item
          $H'$ has all of the trees $H$ had, plus a new tree for each of $x$'s children and a new tree for all but the last call to
          \texttt{CASCADING-CUT}. This gives $t(H') = t(H) + d + c - 1$. \texttt{PISANO-DELETE} calls \texttt{CASCADING-CUT} $c$ times, which reduces
          the number of marks by at least $c-2$. So $m(H') \leq m(H) - c + 2$. Therefore,
          \[ \varphi(H') \leq t(H) + d + c - 1 + 2 m(H) - 2c + 4 = t(H) + 2m(H) + d - c + 3 .\]

        \item
          From parts (b) and (c), we have
          \begin{align*}
            \widehat{c} &\leq O(d+c) + t(H) + 2 m(H) + d - c + 3 - t(H) - 2 m(H) \\
            &= O(d+c) + d - c + 3
          \end{align*}

          Be rescaling the potential function, we can cancel the $O(c)$ and $c$ terms. This leaves an amortized cost that is $O(d)$. Because $d =
          x.degree$ and $D(n) = O(\log n)$, we have an amortized cost of $O(\log n)$, which is the same as for \texttt{FIB-HEAP-DELETE}.
      \end{enumerate}

    \end{solution}

  \item
    \begin{problem}
      Suppose that a sequence of update operations is performed on a persistent red-black tree, $T$, starting from an initially-empty tree. Assume that
      $T$ is implemented using the limited node-copying method. Use the accounting method to prove that the amortized space cost of each update
      operation is $O(1)$. State clearly the invariant you use and the amortized cost that you assign to each update operation.
    \end{problem}

    \begin{solution}

      When updating our tree, new nodes need to be added when a new node is inserted into the tree or when a node is copied because pointers to its
      children are changed (because they are copied or inserted). In the \texttt{RB-TREE-INSERT} and \texttt{RB-TREE-DELETE} operations, a bounded
      number of pointers are changed because these operations only require a bounded number of rotations to restore the red-black property.

      By assigning a bounded number of credits to these operations, we will be able to pay for the new nodes that are possibly created and store credits
      on the nodes where cascading stopped for when they need to be copied in a future operation. An update operation may cascade the entire height of the tree with copying
      nodes, but our allocation of credits will allow this to happen in $O(1)$ space.

      Let us define a node to be full when it has filled its extra pointer slot.

      First, we notice that when a cascade of nodes is being copied, each node that is copied had to originally be empty. The cascade ends when it
      reaches a node that was originally empty, and this node becomes full to end the cascade. Cascades are started when a node has one of its child
      pointers changed, so we must keep track of these changes.

      We next notice that restoring the red-black property after insertion or deletion requires color flips and rotations. Each color flip, only
      changes colors of nodes, so this will not require any node copying. (It is interesting to note that this is the case because we are only
      interested in being able to update the current tree. If we were interested in updating a past tree, we would either have to copy nodes when
      colors are changed or have some method of recoloring a tree.) During a rotation, we can see that a total of 3 pointers are changed. Now let's
      investigate what happens in insertion and deletion more closely.

      During an insertion, a new node is added to the tree, which also creates a new pointer. Then at most 2 rotations must occur to restore the
      red-black property. This causes a total of at most 6 pointer changes. In this case, I will define the amortized cost to be 8; 1 credit will
      pay for the new node inserted, and 7 credits will be placed at the end of the cascades, with some credits potentially being
      unused.

      Deletions are slightly more difficult. First, let's see what happens before the red-black property is restored. If the node to be deleted has no
      children, a single pointer is changed. If the node to be deleted has one child, a single pointer is changed. If the node to be deleted,
      $x$, has two children, we first find the successor, $y$, to $x$. We can then create a new copy of this node and place it in the tree where
      $x$ originally was (can't just change key values when making operations persistent). This changes a single pointer (from $x$'s parent). Then the
      original $y$ must be removed, which changes a single pointer (because $y$ could have at most one child if it is the minimum of $x$'s right
      subtree). So this part of the process requires at most one new node and two pointer changes.

      To restore the red-black property, at most 3 rotations are required. Each of these changes 3 pointers. This will give us an amortized cost of
      12; 1 credit will pay for the potential new node we have, and 11 credits will be placed at the end of the cascades.

      For our analysis, we will define a node to be full if it has used its extra pointer slot. This allows us to use the following invariant:

      \underline{Invariant:} Each full node following an update has at least one credit.
      This invariant is clearly true when the tree is empty.

      Now assume the invariant holds before the current operation.

      If the current operation is an insert, we have 8 credits to allocate. 1 credit is
      used for the new node we are inserting. If the parent of this new node is full, it will be copied with this cost paid for by the credit we
      assumed it has. This copying could cause a cascade of nodes to be copied. Each of these nodes will be full, so the copying will be paid for by
      the credit we assumed each full node has. Now we allocate 1 credit to the newly-filled node at the end of the cascade. To restore the red-black property, we perform color flips and rotations.
      Each of the up to 2 rotations we perform changes 3 pointers. Just as before, these changes could cause a cascade of full nodes being copied
      using the credits they have stored, then we can add a credit to the newly-filled node at the end. Thus we have stored a credit at every full
      node and managed to not run a credit deficit.

      If the current operation is a deletion, the analysis is similar. We have a credit to pay for the new node we may be creating. We change up to
      11 pointers. Each of these changes causes a cascade (possibly of length 0) which we store a credit at the end of. Again, we have maintained the
      invariant without running a credit deficit.

      Thus our invariant holds. We have allocated $O(1)$ credits for each operation, so the total amortized space is $O(1)$ per node.

    \end{solution}

  \item
    \begin{problem}
      Let $S=\{ p_i = (x_i,y_i), 1 \leq i \leq n \}$ be a set of $n$ points in the plane. Assume, for simplicity, that all points have positive and
      distinct $x$- and $y$-coordinates. We would like to pre-process $S$ into a data structure so that the following query can be answered
      efficiently:

      Given any query rectangle $q = [a,b] \times [0,c]$, resting on the $x$-axis, report the $x$-coordinates of the points of $S$ that are contained
      in $q$.

      The desired time bounds are $O(n)$ space, $O(n \log n)$ pre-processing time, and $O(K + \log n)$ query time, where $K$ is the number of reported
      points.

      Do the following:

      \begin{enumerate}
        \item
          Show how to build a data structure for this problem that is based on a persistent red-black tree. Describe briefly the main ideas and give
          pseudocode.

        \item
          Show how to query the structure. Describe briefly the main ideas, give pseudocode, and argue briefly why the query algorithm works.

        \item
          Analyze carefully the space, pre-processing time, and query time of your solution.

      \end{enumerate}

      Note: Assume you have available persistent counterparts to the routines for standard (i.e., non-persistent) red-black trees. Specifically, there
      are available routines \texttt{Insert(tree, key, time)}, \texttt{Delete(tree,key,time)}, \texttt{Search(tree, key, time)} to insert, delete, or
      search a key in a persistent red-black tree at some time instant, as well as a routine \texttt{Access-Range(tree, range, time)} to retrieve from
      a persistent red-black tree the $K$ keys that lie in some range (i.e., interval) at some time instant in $O(K + \log n)$ time. (You may not need
      all of these in your solution.) Assume that the ``limited node copying'' method is used. Treat the persisten red-black tree as a black-box and
      issue calls to the above routines as appropriate.

    \end{problem}

    \begin{solution}

      \begin{enumerate}
        \item
          We can build a data structure by sweeping in the vertical direction. The points must be sorted in increasing $y$-coordinate order. Then
          points will be added in this sorted order to a persistent red-black tree at ``times'' given by the $y$-coordinate of points with the keys
          giving the $x$-values of the points. The following gives pseudocode for the construction of this structure:

          \begin{algorithmic}[1]
            \Function{Create-Tree}{$S$}

            \State Sort $S$ in increasing order of $y_i$
            \State $T \gets \emptyset$
            \For{ $i \gets 1 .. n$ }
            \State \texttt{Insert} $(T,x_i,y_i)$
            \EndFor

            \EndFunction

          \end{algorithmic}

        \item
          Let the rectangle we are interested in be given by $q = [a,b] \times [0,c]$. We can query the data structure created in (a) to solve our
          problem by searching the red-black tree at a time $c$ for values between $a$ and $b$. This will return the correct $x$-values because the
          tree at time $c$ contains all points with $y$-values less than $c$, and searching for keys between $a$ and $b$ restricts the $x$-values of
          points returned to being between $a$ and $b$. Thus, all points returned lie within the rectangle $q$. The following gives pseudocode for
          this querying:

          \begin{algorithmic}[1]

            \Function{Query}{$a,b,c$}

            \State Find largest update time $t$ satisfying $t \leq c$.

            \Return \texttt{Access-Range}($T, [a,b], t$)

            \EndFunction

          \end{algorithmic}

        \item
          By the analysis in problem 4, the space required for the red-black tree we construct is $O(n)$ because each update of the tree has an
          amortized cost of $O(1)$. The querying time is $O(K + \log n)$ because it only calls on the persistent \texttt{Access-Range} operation,
          which we have assumed runs in $O(K + \log n)$ time, and searches for an update time in $n$ values, which can be done in $\log n$ time using
          a binary search tree.

          The pre-processing step requires building the persistent red-black tree. This starts by sorting the points in $S$ based on increasing
          $y$-coordinates. This step takes $O(n \log n)$ time. Then we loop over each point and add its $x$-value to the tree. The update operation is
          assumed to take $O(\log n)$ time for the persistent red-black tree. Therefore, this step takes $O(n \log n)$ time overall, giving the
          desired time bound of $O(n \log n)$.

      \end{enumerate}

    \end{solution}

\end{enumerate}
\end{document}


