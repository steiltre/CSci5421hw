%        File: hw3.tex
%     Created: Mon Oct 17 04:00 PM 2016 C
% Last Change: Mon Oct 17 04:00 PM 2016 C
%

\documentclass[a4paper]{article}

\title{CSci 5641 Homework 3 }
\date{10/26/16}
\author{Trevor Steil}

\usepackage{amsmath}
\usepackage{amsthm}
\usepackage{amssymb}
\usepackage{esint}

\newtheorem{theorem}{Theorem}[section]
\newtheorem{corollary}{Corollary}[section]
\newtheorem{proposition}{Proposition}[section]
\newtheorem{lemma}{Lemma}[section]
\newtheorem*{claim}{Claim}
\newtheorem*{problem}{Problem}
%\newtheorem*{lemma}{Lemma}
\newtheorem{definition}{Definition}[section]

\newcommand{\R}{\mathbb{R}}
\newcommand{\N}{\mathbb{N}}
\newcommand{\C}{\mathbb{C}}
\newcommand{\Z}{\mathbb{Z}}
\newcommand{\supp}[1]{\mathop{\mathrm{supp}}\left(#1\right)}
\newcommand{\lip}[1]{\mathop{\mathrm{Lip}}\left(#1\right)}
\newcommand{\curl}{\mathrm{curl}}
\newcommand{\la}{\left \langle}
\newcommand{\ra}{\right \rangle}
\renewcommand{\vec}[1]{\mathbf{#1}}

\newenvironment{solution}{\emph{Solution.}}

\begin{document}
\maketitle

\begin{enumerate}
  \item
    \begin{problem}
      Assume that you are given a set $P = \{p_1 < p_2 < \dots < p_n \}$ of points on the real line; the distance between consecutive points can be
      arbitrary. We would like to determine the smallest number of non-overlapping intervals, each of length 1, to place on the real line so that each
      point of $P$ is contained in some interval. (Here ``non-overlapping intervals'' means that the intervals have no point in common.)

      Describe briefly, in words, a greedy algorithm for this problem (pseudocode is not required). Prove it correct using the 2-step method, i.e.,
      state and establish carefully the greedy choice and the optimal substructure properties.
    \end{problem}

    \begin{solution}

    \end{solution}

  \item
    \begin{problem}
      Let $A$ and $B$ be sequences of $n$ positive integers each. You are allowed to re-order $A$ and $B$ as you wish. Let $A = a_1, a_2, \dots , a_n$
      and $B=b_1, b_2, \dots, b_n$ after the re-ordering. The goal is to come up with an ordering which maximizes $\Pi_{i=1}^n a_i^{b_i}$.

      Describe briefly, in words, a greedy algorithm for this problem (pseudocode is not required). Prove it correct using the 2-step method, i.e.,
      state and establish carefully the greedy choice and the optimal substructure properties.
    \end{problem}

    \begin{solution}

    \end{solution}

  \item
    \begin{problem}
      Ex. 23.1-5, p. 629
    \end{problem}

    \begin{solution}

    \end{solution}

  \item
    \begin{problem}
      Ex. 23.1-10, p. 630
    \end{problem}

    \begin{solution}

    \end{solution}

  \item

  \item

\end{enumerate}

\end{document}


