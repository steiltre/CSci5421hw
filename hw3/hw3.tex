%        File: hw3.tex
%     Created: Mon Oct 17 04:00 PM 2016 C
% Last Change: Mon Oct 17 04:00 PM 2016 C
%

\documentclass[a4paper]{article}

\title{CSci 5421 Homework 3}
\date{10/26/16}
\author{Trevor Steil}

\usepackage{amsmath}
\usepackage{amsthm}
\usepackage{amssymb}
\usepackage{esint}

\newtheorem{theorem}{Theorem}[section]
\newtheorem{corollary}{Corollary}[section]
\newtheorem{proposition}{Proposition}[section]
\newtheorem{lemma}{Lemma}[section]
\newtheorem*{claim}{Claim}
\newtheorem*{problem}{Problem}
%\newtheorem*{lemma}{Lemma}
\newtheorem{definition}{Definition}[section]

\newcommand{\R}{\mathbb{R}}
\newcommand{\N}{\mathbb{N}}
\newcommand{\C}{\mathbb{C}}
\newcommand{\Z}{\mathbb{Z}}
\newcommand{\supp}[1]{\mathop{\mathrm{supp}}\left(#1\right)}
\newcommand{\lip}[1]{\mathop{\mathrm{Lip}}\left(#1\right)}
\newcommand{\curl}{\mathrm{curl}}
\newcommand{\la}{\left \langle}
\newcommand{\ra}{\right \rangle}
\renewcommand{\vec}[1]{\mathbf{#1}}

\newenvironment{solution}{\emph{Solution.}}

\begin{document}
\maketitle

\begin{enumerate}
  \item
    \begin{problem}
      Assume that you are given a set $P = \{p_1 < p_2 < \dots < p_n \}$ of points on the real line; the distance between consecutive points can be
      arbitrary. We would like to determine the smallest number of non-overlapping intervals, each of length 1, to place on the real line so that each
      point of $P$ is contained in some interval. (Here ``non-overlapping intervals'' means that the intervals have no point in common.)

      Describe briefly, in words, a greedy algorithm for this problem (pseudocode is not required). Prove it correct using the 2-step method, i.e.,
      state and establish carefully the greedy choice and the optimal substructure properties.
    \end{problem}

    \begin{solution}

      We begin by sorting $P$ in increasing order. We will construct our collection of intervals by choosing the smallest value in $P$ which is not already contained in an interval. Let this
      value be $p_i \in P$. We will then add the interval $[p_i, p_i + 1]$ to our collection of intervals. We must show that this algorithm has the
        Greedy Choice Property and Optimal Substructures.

        \begin{claim} (Greedy Choice Property) There is an optimal collection of non-overlapping intervals for $P$ which contains $\left[p_1, p_1 +
          1 \right]$.
        \end{claim}

        \begin{proof}

          Let $B = \{ \left[i_1, i_1 + 1\right], \left[i_2, i_2 + 1\right], \dots, \left[i_k, i_k+1\right] \}$ with $i_j \leq i_{j+1}$ for $1 \leq j
          \leq k-1$ be an optimal set of intervals for $P$.
          If $\left[p_1, p_1 + 1 \right] \in B$, then we are done.

          Assume $\left[p_1, p_1 + 1\right] \not \in B$. Then we must have $i_1 < p_1$ because $p_1$ must be in some interval. We can replace $[ i_1,
            i_1 + 1]$ with $[p_1, p_1 + 1]$. Because $p_1 < p_2< \dots < p_n$,
              \[ B' = \left(B \setminus \{ [i_1, i_1 + 1] \} \right) \cup \left\{ [p_1, p_1+1] \right\} = \left\{ [p_1, p_1+1], [i_2, i_2+1], \dots, [i_2, i_2+1]
              \right\} \]
          is a set of intervals that contains $p_1, p_2, \dots, p_n$. The issue with replacing this interval is that we may have created $B'$ in such
          a way that it contains overlapping intervals. If $B'$ doesn't have overlapping intervals, it will be an optimal solution because it has the
          same number of intervals as $B$.

          If $B'$ has an overlapping interval, the only possibility is ${ [p_1, p_1 +1] \cap [i_2, i_2 +1] \neq \emptyset }$ because $B$ did not contain
          overlapping intervals. This means $i_2 \leq p_1 + 1$. Let $j$ be the smallest integer such that $p_j \not \in [p_1, p_1 + 1]$. Because
          $p_{i-1} \in [p_1, p_1 + 1]$, $p_{i} \in [p_i, p_i + 1]$, and $i_2 < p_i$
            \[ B'' = \left(B' \setminus [i_2, i_2 + 1] \right) \cup \{ [p_1 + 1, p_1 + 2] \} \]
          must contain $p_1, p_2, \dots, p_n$.

          By construction, we swapped single intervals in $B$ for a single desired interval, so $B''$ contains the same number of intervals as $B$.
          This may cause another set of overlapping intervals. In this case, we repeat the same replacement procedure until we do not have overlapping
          intervals. $B$ contains only $k$ intervals, so this process must terminate after a finite number of steps.

          This will result in a set of
          intervals that contains $[p_1, p_1 + 1]$, contains no overlapping intervals, and has the same number of intervals as an optimal solution.
            Therefore, we get an optimal solution with the desired property.

        \end{proof}

        \begin{claim} (Optimal Substructure Property) Let $P' = P \setminus \{ p_i : p_i < p_1 + 1 \} = \{ p_i, p_{i+1}, \dots, p_n \}$ for some $i$. Let $B$ be an optimal solution constructed as
          above. Then ${ B' = B \setminus \{ [p_1, p_1+1] \} }$ is an optimal solution for $P'$.
        \end{claim}

        \begin{proof}

          By definition, $P'$ contains all points not contained in the first interval of the solution we are constructing. Therefore, $B'$ is a
          solution for $P'$.

          Assume $B'$ is not an optimal solution for $P'$. Then we can find an optimal solution $C$ with $|C| < |B'|$ (where the cardinality is
          understood to be the finite number of intervals contained in each set). As in the proof of the greedy choice property, we can modify $C$ to
          be an optimal solution which contains the interval $[ p_i, p_i +1].$ Then $C \cup \{ [p_1, p_1 + 1] \}$ contains all of the points in $P$.
          By modifying $C$ to contain $[ p_i, p_i + 1]$, we know $C$ does not contain any overlapping intervals. Also,
            \[ | C \cup \{ [ p_i, p_i + 1 ] \} | < |B'| + 1 = |B|. \]
          This is a contradiction because $B$ was assumed to be an optimal solution. Therefore, $B'$ is an optimal solution for $P'$, and we have the optimal substructure property.

        \end{proof}

        Because we have the Greedy Choice Property and the Optimal Substructure Property, our algorithm will return an optimal solution. This
        algorithm requires scanning through $P$ a single time to determine which intervals need to be included, so the running time is dominated by
        the sort which takes time $O(n \log n)$.

    \end{solution}

  \item
    \begin{problem}
      Let $A$ and $B$ be sequences of $n$ positive integers each. You are allowed to re-order $A$ and $B$ as you wish. Let $A = a_1, a_2, \dots , a_n$
      and $B=b_1, b_2, \dots, b_n$ after the re-ordering. The goal is to come up with an ordering which maximizes $\Pi_{i=1}^n a_i^{b_i}$.

      Describe briefly, in words, a greedy algorithm for this problem (pseudocode is not required). Prove it correct using the 2-step method, i.e.,
      state and establish carefully the greedy choice and the optimal substructure properties.
    \end{problem}

    \begin{solution}

      We are going to order $A$ and $B$ in decreasing order. By pairing the largest bases with the largest exponents, we will create the optimal
      solution.

      \begin{claim} (Greedy Choice Property)
        Let $A$ and $B$ be ordered to give the minimum value for $\Pi_{i=1}^n a_i^{b_i}$. There are reorderings $\tilde{A}$ and $\tilde{B}$ of $A$ and $B$ such that $a_1 =
        \max (A)$ and $b_1 = \max (B)$ which gives an optimal solution.
      \end{claim}

      \begin{proof}

        Let $a_i = \max(A)$ and $b_j = \max(B)$. Define
        \[ \tilde{A} = a_i, a_2, \dots, a_{i-1}, a_1, a_{i+1}, \dots, a_n \]
        and
        \[ \tilde{B} = b_j, b_2, \dots, b_{j-1}, b_1, b_{j+1}, \dots, b_n .\]

        Without loss of generality, we will assume $b_1 \geq b_i$. If this were not the case, we would also swap $b_1$ and $b_i$ in the definition of
        $\tilde{B}$.

        First, we will show that the portion of the product that has been modified has been made larger. That is $a_i^{b_j} a_1^{b_i} a_j^{b_1} \geq
        a_1^{b_1} a_i^{b_i} a_j^{b_j}$. We have
        \begin{align*}
          \frac{a_i^{b_j} a_1^{b_i} a_j^{b_1}}{a_1^{b_1} a_i^{b_i} a_j^{b_j}} &= a_i^{b_j - b_i} a_1^{b_i - b_1}
          a_j^{b_1 - b_j} \\
          &\geq a_i^{b_j - b_i} a_1^{b_i - b_1} a_i^{b_1 - b_j} \quad \parbox{5cm}{because $b_1 - b_j \leq 0$ and $a_i \geq a_j$} \\
          &= a_i^{b_1 - b_i} a_1^{b_i - b_1} \\
          &\geq a_i^{b_1 - b_i} a_i^{b_i - b_1} \quad \parbox{5cm}{because $b_i - b_1 \leq 0$ and $a_i \geq a_1$} \\
          &= 1
        \end{align*}

        Therefore,
        \begin{equation}\label{ineq:greedy1}
          a_i^{b_j} a_1^{b_i} a_j^{b_1} \geq a_1^{b_1} a_i^{b_i} a_j^{b_j} .
        \end{equation}

        The entries corresponding to the indices $1, i$, and $j$, are the only values in the product that were modified. Let
        \[ x = \prod\limits_{\substack{k=2, \\k \neq i \\ k \neq j}}^n a_k^{b_k} \]
        Also, let $\tilde{A} = \tilde{a}_1 , \tilde{a}_2, \dots, \tilde{a}_n$ and $\tilde{B} = \tilde{b}_1, \tilde{b}_2, \dots, \tilde{b}_n$ be our
        reorderings of $A$ and $B$, i.e., $\tilde{a}_1 = a_i, \tilde{a}_i = a_1, \tilde{b}_1 = b_j,$ and $\tilde{b}_j = b_1$ with all other $a$'s and
        $b$'s remaining unchanged.

        Then
        \begin{align*}
          \prod\limits_{k=1}^n \tilde{a}_k^{\tilde{b}_k} &= a_i^{b_j} a_1^{b_i} a_j^{b_1} x \\
          &\geq a_1^{b_1} a_i^{b_i} a_j^{b_j} x \quad \parbox{5cm}{by \eqref{ineq:greedy1}} \\
          &= \prod\limits_{k=1}^n a_k^{b_k}
        \end{align*}

        By the optimality of $A$ and $B$, we have
        \[ \prod\limits_{k=1}^n \tilde{a}_k^{\tilde{b}_k} \leq \prod\limits_{k=1}^n a_k^{b_k} .\]
        Therefore,
        \[ \prod\limits_{k=1}^n \tilde{a}_k^{\tilde{b}_k} = \prod\limits_{k=1}^n a_k^{b_k} .\]

        Thus, $\tilde{A}$ and $\tilde{B}$ define an optimal solution with $a_1$ and $b_1$ being the maximal elements of $A$ and $B$, respectively.

      \end{proof}

      \begin{claim} (Optimal Substructure Property)
        Let $A$ and $B$ be ordered to maximize $\prod\limits_{k=1}^n a_k^{b_k}$. Then $A' = a_2,\dots, a_n$ and $B' = b_2,\dots, b_n$ are ordered to
        maximize $\prod\limits_{k=2}^n a_k^{b_k}$.
      \end{claim}

      \begin{proof}

        This argument is part of what was used to prove the Greedy Choice Property. Assume there is a pair of reorderings $\tilde{A}' =
        \tilde{a}_2,\dots, \tilde{a}_n$ and $\tilde{B}' = \tilde{b}_2, \dots, \tilde{b}_n$ such that
        \[ \prod\limits_{k=2}^n \tilde{a}_k^{\tilde{b}_k} > \prod\limits_{k=2}^n a_k^{b_k} .\]
        Then
        \[ a_1^{b_1} \prod\limits_{k=2}^n \tilde{a}_k^{\tilde{b}_k} > a_1^{b_1} \prod\limits_{k=2}^n a_k^{b_k} .\]
        Thus $\tilde{A} = a_1, \tilde{a}_2, \dots , \tilde{a}_n$ and $\tilde{B} = b_1, \tilde{b}_2, \dots, \tilde{b}_n$ is a reordering with a greater
        maximum, which contradicts the optimality of $A$ and $B$.

      \end{proof}

      Because the Greedy Choice Property and the Optimal Substructure Property hold, this algorithm will yield an optimal solution. This algorithm
      only involves sorting $A$ and $B$, so it will have a running time of $O(n \log n)$.

    \end{solution}

    \pagebreak

  \item
    \begin{problem}
      Ex. 23.1-5, p. 629

      Let $e$ be a maximum-weight edge on some cycle of connected graph $G = (V,E)$. Prove that there is a minimum spanning tree of $G' = (V,E
      \setminus \{e\})$ that is also a minimum spanning tree of $G$. That is, there is a minimum spanning tree of $G$ that does not include $e$.
    \end{problem}

    \begin{solution}

      Let $T$ be a minimum spanning tree for $G$. If $T$ does not contain $e$, we are done. Assume that $T$ does contain $e$. $T$ is a tree, so
      removing $e$ partitions $T$ into two sets. Therefore, we have a cut $(S, V \setminus S)$ which the edge $e$ crosses and which respects $T
      \setminus \{e\}$.

      Because $e$ is part of a
      cycle, there is another edge $\overline{e}$ that crosses this cut. We can find an edge ${e'}$ which
      crosses the cut with $w(e') \leq w(d)$ for any edge $d$ crossing the cut. By the maximality of $w(e)$ in its cycle, we can take $e' \neq e$.
      Notice that $e' \not \in T$ because $T$ is a tree.

      $T \setminus \{e\}$ is clearly contained in the minimal spanning tree $T$. We have defined $e'$ to be a light edge. Therefore, by
      Theorem 23.1, $e'$ is a safe edge for $T \setminus \{e\}$. Therefore, $\left( T \setminus \{e\} \right) \cup \{e'\}$ is contained in some
      minimal spanning tree $T'$. In fact,
      \begin{align*}
        \left| \left( T \setminus \{e\} \right) \cup \{e'\} \right| &= |T| \\
        &= |V| - 1
      \end{align*}
      Therefore, the minimal spanning tree containing $\left( T \setminus \{e\} \right) \cup \{e'\}$ is
      \[ T'= \left( T \setminus \{e\} \right) \cup \{e'\} .\]
      Because we chose $e' \neq e$ with $e' \not \in T$, $T' \neq T$. Thus, we have a minimal spanning tree, $T'$, which does not contain $e$.

    \end{solution}

  \item
    \begin{problem}
      Ex. 23.1-10, p. 630

      Given a graph $G$ and a minimum spanning tree $T$, suppose that we decrease the weight of one of the edges in $T$. Show that $T$ is still a
      minimum spanning tree for $G$. More formally, let $T$ be a minimum spanning tree for $G$ with edge weights given by weight function $w$. Choose
      one edge $(x,y) \in T$ and a positive number $k$, and define the weight function $w'$ by
      \[ w'(u,v) =
        \begin{cases} w(u,v) &\text{if } (u,v) \neq (x,y) , \\
          w(x,y) - k &\text{if } (u,v) = (x,y) .
        \end{cases}
      \]
      Show that $T$ is a minimum spanning tree for $G$ with edge weights given by $w'$.

    \end{problem}

    \begin{solution}

      Assume toward a contradiction that $T$ is not a minimum spanning tree for $G$ with edge weights given by $w'$. That means we can find a spanning
      tree $T'$ with $w'(T) > w'(T')$. We know that $(x,y) \in T'$ or $(x,y) \not \in T'$.

      If $(x,y) \in T'$, then we know
      \begin{align*}
        w(T') - k &= w'(T') \quad \parbox{5cm}{by definition} \\
        &< w'(T) \quad \parbox{5cm}{by assumption} \\
        &= w(T) - k \quad \parbox{5cm}{because $(x,y) \in T$}
      \end{align*}

      Therefore, we know $w(T') < w(T)$, contradicting the minimality of $w(T)$.

      If $(x,y) \not \in T'$, then
      \begin{align*}
        w(T) &\leq w(T') \quad \parbox{5cm}{by minimality of $w(T)$} \\
        &= w'(T') \quad \parbox{5cm}{because $(x,y) \not \in T'$} \\
        &< w'(T) \quad \parbox{5cm}{by minimality of $w'(T')$} \\
        &= w(T) - k \quad \parbox{5cm}{because $(x,y) \in T$}
      \end{align*}

      Therefore, $w(T) < w(T) - k$ where $k$ is a positive number, which is a contradiction.

      Because we have reached a contradiction whether we assume $(x,y) \in T$ or $(x,y) \not \in T$, $T$ must be a minimal spanning tree for $G$ with
      edge weights given by $w'$.

    \end{solution}

  \item
    \begin{problem}
      Ex. 16.4-4, p. 443

      Let $S$ be a finite set and let $S_1, S_2, \dots, S_k$ be a partition of $S$ into nonempty disjoint subsets. Define the structure $(S,
      \mathcal{I})$ by the condition that
      \[ \mathcal{I} = \{ A \subseteq S : |A \cap S_i| \leq 1 \text{ for } i=1,2,\dots, k\}. \]
      Show that $(S,
      \mathcal{I})$ is a matroid. That is, the set of all sets $A$ that contain at most one member of each subset in the partition determines the
      independent sets of a matroid.

    \end{problem}

    \begin{proof}

    Take a set $B \in \mathcal{I}$ and $A \subset B$. $A \cap S_i \subset B \cap S_i$ for all $i$. Therefore,
    \[ \left| A \cap S_i \right| \leq \left| B \cap S_i \right| \leq 1 \]
    for $1 \leq i \leq k$. Then $A \in \mathcal{I}$, and $\mathcal{I}$ is hereditary.

    Now take $A, B \in \mathcal{I}$ with $|A| < |B|$. Because $|A| < |B|$, there is some $j$ such that $|B \cap S_j| = 1$ and $|A \cap S_j| = 0$. Let
    $x = B \cap S_j$. Define $A' = A \cup \{x\}$. Then $A' \cap S_j = \{x\}$. Therefore, $A'$ satisfies $|A' \cap S_i| \leq 1$ for all $i$. Thus, $A' \in \mathcal{I}$ and $M$ satisfies the
    exchange property.

    Because $S$ is assumed to be finite, $\mathcal{I}$ is hereditary, and $M$ satisfies the exchange property, we know $M$ is a matroid.

    \end{proof}

    \pagebreak

  \item
    \begin{problem}

      Let $M = (S, \mathcal{I})$ be a matroid. Let $H$ be any given subset of $S$ and define
      \[ \mathcal{I}' = \{ C | C \in \mathcal{I} \text{ and } C \cap H = \emptyset \} . \]
      Prove that $M' = ( S, \mathcal{I}')$ is a matroid by showing that the defining properties of a matroid hold.

    \end{problem}

    \begin{proof}

      Let $B \in \mathcal{I}'$ and $A \subset B$. Because $\mathcal{I}$ is hereditary and $B \in \mathcal{I}$, $A \in \mathcal{I}$ as well. We also know
      \[ A \cap H \subset B \cap H = \emptyset .\]
      Therefore, $A \cap H = \emptyset$. Thus, $A \in \mathcal{I}'$ and $\mathcal{I}'$ is hereditary.

      Now let $A, B \in \mathcal{I'}$ with $|A| < |B|$. Because $M$ satisfies the exchange property, there is some $x \in B$ such that $A
      \cup \{x\} \in \mathcal{I}$. Also
      \begin{align*}
        \left( A \cup \{x\} \right) \cap H &\subseteq \left( A \cup B \right) \cap H \\
        &= \left( A \cap H \right) \cup \left( B \cap H \right) \\
        &= \emptyset \cup \emptyset \\
        &= \emptyset
      \end{align*}
      Thus, $M'$ satisfies the exchange property.

      Because $M$ is a matroid, we know $S$ is finite. Therefore, $M'$ is a matroid.

    \end{proof}

\end{enumerate}

\end{document}


