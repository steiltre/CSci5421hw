%        File: hw3.tex
%     Created: Mon Oct 17 04:00 PM 2016 C
% Last Change: Mon Oct 17 04:00 PM 2016 C
%

\documentclass[a4paper]{article}

\title{CSci 5641 Homework 3 }
\date{10/26/16}
\author{Trevor Steil}

\usepackage{amsmath}
\usepackage{amsthm}
\usepackage{amssymb}
\usepackage{esint}

\newtheorem{theorem}{Theorem}[section]
\newtheorem{corollary}{Corollary}[section]
\newtheorem{proposition}{Proposition}[section]
\newtheorem{lemma}{Lemma}[section]
\newtheorem*{claim}{Claim}
\newtheorem*{problem}{Problem}
%\newtheorem*{lemma}{Lemma}
\newtheorem{definition}{Definition}[section]

\newcommand{\R}{\mathbb{R}}
\newcommand{\N}{\mathbb{N}}
\newcommand{\C}{\mathbb{C}}
\newcommand{\Z}{\mathbb{Z}}
\newcommand{\supp}[1]{\mathop{\mathrm{supp}}\left(#1\right)}
\newcommand{\lip}[1]{\mathop{\mathrm{Lip}}\left(#1\right)}
\newcommand{\curl}{\mathrm{curl}}
\newcommand{\la}{\left \langle}
\newcommand{\ra}{\right \rangle}
\renewcommand{\vec}[1]{\mathbf{#1}}

\newenvironment{solution}{\emph{Solution.}}

\begin{document}
\maketitle

\begin{enumerate}
  \item
    \begin{problem}
      Assume that you are given a set $P = \{p_1 < p_2 < \dots < p_n \}$ of points on the real line; the distance between consecutive points can be
      arbitrary. We would like to determine the smallest number of non-overlapping intervals, each of length 1, to place on the real line so that each
      point of $P$ is contained in some interval. (Here ``non-overlapping intervals'' means that the intervals have no point in common.)

      Describe briefly, in words, a greedy algorithm for this problem (pseudocode is not required). Prove it correct using the 2-step method, i.e.,
      state and establish carefully the greedy choice and the optimal substructure properties.
    \end{problem}

    \begin{solution}

      We will construct our collection of intervals by choosing the smallest value in $P$ which is not already contained in an interval. Let this
      value be $p_i \in P$. We will then add the interval $[p_i, p_i + 1)$ to our collection of intervals. We must show that this algorithm has the
        Greedy Choice Property and Optimal Substructures.

        \begin{claim} (Greedy Choice Property) There is an optimal collection of non-overlapping intervals for $P$ which contains $\left[p_1, p_1 +
          1 \right)$.
        \end{claim}

        \begin{proof}

          Let $B = \{ \left[i_1, i_1 + 1\right), \left[i_2, i_2 + 1\right), \dots, \left[i_k, i_k+1\right) \}$ with $i_i \leq i_{i+1}$ for $1 \leq i
          \leq k-1$ be an optimal set of intervals for $P$.
          If $\left[p_1, p_1 + 1 \right) \in B$, then we are done.

          Assume $\left[p_1, p_1 + 1\right) \not \in B$. Then we must have $i_1 < p_1$ because $p_1$ must be in some interval. We can replace $[ i_1,
            i_1 + 1)$ with $[p_1, p_1 + 1)$. Because $p_1 < p_2< \dots < p_n$,
              \[ B' = \left(B \setminus \{ [i_1, i_1 + 1) \} \right) \cup \left\{ [p_1, p_1+1) \right\} = \left\{ [p_1, p_1+1), [i_2, i_2+1), \dots, [i_2, i_2+1)
              \right\} \]
          is a set of intervals that contains $p_1, p_2, \dots, p_n$. The issue with replacing this interval is that we may have created $B'$ in such
          a way that it contains overlapping intervals. If $B'$ doesn't have overlapping intervals, it will be an optimal solution because it has the
          same number of intervals as $B$.

          If $B'$ has an overlapping interval, the only possibility is ${ [p_1, p_1 +1) \cap [i_2, i_2 +1) \neq \emptyset }$ because $B$ did not contain
            overlapping intervals. This means $i_2 < p_1 + 1$. In this case, we can replace $[ i_2, i_2 + 1)$ with $[p_1 + 1, p_1 + 2)$. Because
              $[i_2, i_2 + 1) \subset [p_1, p_1 + 1) \cup [p_1 + 1, p_1 + 2) = [p_1, p_1 + 2)$, we must have
                \[ B'' = \left(B' \setminus [i_2, i_2 + 1) \right) \cup \{ [p_1 + 1, p_1 + 2) \} \]
              must contain $p_1, p_2, \dots, p_n$. We also see that by construction, $B''$ contains the same number of intervals as $B$.
          This may cause another set of overlapping intervals. In this case, we repeat the same replacement procedure until we do not have overlapping
          intervals. $B$ contains only $k$ intervals, so this process must terminate after a finite number of steps. This will result in a set of
          intervals that contains $[p_1, p_1 + 1)$, contains no overlapping intervals, and has the same number of intervals as an optimal solution.
            Therefore, we get an optimal solution with the desired property.

        \end{proof}

        \begin{claim} (Optimal Substructure Property) Let $P' = P \setminus \{ p_i : p_i < p_1 + 1 \} = \{ p_i, p_{i+1}, \dots, p_n \}$ for some $i$. Let $B$ be an optimal solution constructed as
          above. Then ${ B' = B \setminus \{ [p_1, p_1+1) \} }$ is an optimal solution for $P'$.
        \end{claim}

        \begin{proof}

          By definition, $P'$ contains all points not contained in the first interval of the solution we are constructing. Therefore, $B'$ is a
          solution for $P'$.

          Assume $B'$ is not an optimal solution for $P'$. Then we can find an optimal solution $C$ with $|C| < |B'|$ (where the cardinality is
          understood to be the finite number of intervals contained in each set). As in the proof of the greedy choice property, we can modify $C$ to
          be an optimal solution which contains the interval $[ p_i, p_i +1).$ Then $C \cup \{ [p_1, p_1 + 1) \}$ contains all of the points in $P$.
          By modifying $C$ to contain $[ p_i, p_i + 1)$, we know $C$ does not contain any overlapping intervals. Also,
            \[ | C \cup \{ [ p_i, p_i + 1 ) \} | < |B'| + 1 = |B|. \]
          This is a contradiction because $B$ was assumed to be an optimal solution. Therefore, we have the optimal substructure property.

        \end{proof}

        Because we have the Greedy Choice Property and the Optimal Substructure Property, our algorithm will return an optimal solution.

    \end{solution}

  \item
    \begin{problem}
      Let $A$ and $B$ be sequences of $n$ positive integers each. You are allowed to re-order $A$ and $B$ as you wish. Let $A = a_1, a_2, \dots , a_n$
      and $B=b_1, b_2, \dots, b_n$ after the re-ordering. The goal is to come up with an ordering which maximizes $\Pi_{i=1}^n a_i^{b_i}$.

      Describe briefly, in words, a greedy algorithm for this problem (pseudocode is not required). Prove it correct using the 2-step method, i.e.,
      state and establish carefully the greedy choice and the optimal substructure properties.
    \end{problem}

    \begin{solution}

      We are going to order $A$ and $B$ in decreasing order.

    \end{solution}

  \item
    \begin{problem}
      Ex. 23.1-5, p. 629

      Let $e$ be a maximum-weight edge on some cycle of connected graph $G = (V,E)$. Prove that there is a minimum spanning tree of $G' = (V,E
      \setminus \{e\})$ that is also a minimum spanning tree of $G$. That is, there is a minimum spanning tree of $G$ that does not include $e$.
    \end{problem}

    \begin{solution}

    \end{solution}

  \item
    \begin{problem}
      Ex. 23.1-10, p. 630

      Given a graph $G$ and a minimum spanning tree $T$, suppose that we decrease the weigth of one of the edges in $T$. Show that $T$ is still a
      minimum spanning tree for $G$. More formally, let $T$ be a minimum spanning tree for $G$ with edge weights given by weight functions $w$. Choose
      one edge $(x,y) \in T$ and a positive number $k$, and define the weight function $w'$ by
      \[ w'(u,v) =
        \begin{cases} w(u,v) &\text{if } (u,v) \neq (x,y) , \\
          w(x,y) - k &\text{if } (u,v) = (x,y) .
        \end{cases}
      \]
      Show that $T$ is a minimum spanning tree for $G$ with edge weights given by $w'$.

    \end{problem}

    \begin{solution}

    \end{solution}

  \item
    \begin{problem}
      Ex. 16.4-4, p. 443

      Let $S$ be a finite set and let $S_1, S_2, \dots, S_k$ be a partition of $S$ into nonempty disjoint subsets. Define the sturcture $(S,
      \mathcal{I})$ by the condition that
      \[ \mathcal{I} = \{ A \subseteq S : |A \cap S_i| \leq 1 \text{ for } i=1,2,\dots, k\}. \]
      Show that $(S,
      \mathcal{I})$ is a matroid. That is, the set of all sets $A$ that contain at most one member of each subset in the partition determines the
      independent sets of a matroid.

    \end{problem}

    \begin{solution}

    \end{solution}

  \item
    \begin{problem}

      Let $M = (S, \mathcal{I})$ be a matroid. Let $H$ be any given subset of $S$ and define
      \[ \mathcal{I}' = \{ C | C \in \mathcal{I} \text{ and } C \cap H = \emptyset \} . \]
      Prove that $M' = ( S, \mathcal{I}')$ is a matroid by showing that the defining properties of a matroid hold.

    \end{problem}

    \begin{solution}

    \end{solution}

\end{enumerate}

\end{document}


