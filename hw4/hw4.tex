%        File: hw4.tex
%     Created: Wed Oct 26 04:00 PM 2016 C
% Last Change: Wed Oct 26 04:00 PM 2016 C
%

\documentclass[a4paper]{article}

\title{CSci 5421 Homework 4 }
\date{11/9/16}
\author{Trevor Steil}

\usepackage{amsmath}
\usepackage{amsthm}
\usepackage{amssymb}
\usepackage{esint}

\newtheorem{theorem}{Theorem}[section]
\newtheorem{corollary}{Corollary}[section]
\newtheorem{proposition}{Proposition}[section]
\newtheorem{lemma}{Lemma}[section]
\newtheorem*{claim}{Claim}
\newtheorem*{problem}{Problem}
%\newtheorem*{lemma}{Lemma}
\newtheorem{definition}{Definition}[section]

\newcommand{\R}{\mathbb{R}}
\newcommand{\N}{\mathbb{N}}
\newcommand{\C}{\mathbb{C}}
\newcommand{\Z}{\mathbb{Z}}
\newcommand{\supp}[1]{\mathop{\mathrm{supp}}\left(#1\right)}
\newcommand{\lip}[1]{\mathop{\mathrm{Lip}}\left(#1\right)}
\newcommand{\curl}{\mathrm{curl}}
\newcommand{\la}{\left \langle}
\newcommand{\ra}{\right \rangle}
\renewcommand{\vec}[1]{\mathbf{#1}}

\newenvironment{solution}{\emph{Solution.}}

\begin{document}
\maketitle
\begin{enumerate}
  \item
    \begin{problem}

      Recall the problem of task scheduling with deadlines and penalties (Sec. 16.5), which was discussed in class. This was solved using the generic
      greedy algorithm for matroid optimization (Sec. 16.4). A key step in the algorithm is to test at each iteration whether a given set, $A$, of
      tasks is independent, i.e., whether there exists a schedule in which no task of $A$ is late. This can be accomplished by using the following
      fact: ``$A$ is independent if and only if $N_t(A) \leq t$ for $t=0,1,\dots,n$.'' Here $N_t(A)$ is the number of tasks in $A$ with deadline at
      most $t$.

      Give an algorithm that tests if $A$ is independent in $\Theta(|A|)$ time. Your answer should include a brief explanation of the key ideas behind
      your approach, pseudocode, and run-time analysis. (Note that a running time of $\Theta(n)$ is easy to achieve but is too large when $|A| \ll n$.)

      Hint: Consider limiting the range of values of $t$ that need to be checked in the fact stated above. Be sure to justify your reasoning.

    \end{problem}

  \item

  \item

  \item

  \item

  \item

  \item

\end{enumerate}
\end{document}


